\documentclass[RNAAS]{aastex631}
\usepackage{hyperref}
\let\tablenum\relax
\usepackage{siunitx}

%% Define new commands here
\newcommand\latex{La\TeX}

\graphicspath{{./}{figures/}}
\shortauthors{Lucas et al.}

\begin{document}

\title{Yeehaw: A Roundup of Probabilistic Transit Modeling Packages}

%% Note that the corresponding author command and emails has to come
%% before everything else. Also place all the emails in the \email
%% command instead of using multiple \email calls.
\correspondingauthor{Miles Lucas}
\email{mdlucas@hawaii.edu}

\author[0000-0001-6341-310X]{Miles Lucas}
\affiliation{Institute for Astronomy, University of Hawai'i \\
2680 Woodlawn Dr, Honolulu, HI 96822, USA}

%% Note that RNAAS manuscripts DO NOT have abstracts.
%% See the online documentation for the full list of available subject
%% keywords and the rules for their use.
\keywords{}

%% Start the main body of the article. If no sections in the 
%% research note leave the \section call blank to make the title.
\section{} 

The method of transit photometry has proven indispensable in the detection and characterization of thousands of exoplanets over the past few decades \citep{2007prpl.conf..701C,2009IAUS..253...99W,2010exop.book...55W,2010trex.book.....H}. Transits are particularly useful for determining relative radii \citep{2019A&A...623A.137H}, constraining orbital inclination \citep{2003ApJ...585.1038S}, and the planet's orbital ephemeris. These quantities all have to be inferred using a computed model of the planetary transit. Model accuracy directly affects the inference of these parameters and a recent study by \citet{2020AJ....159..123A} has improved numerical accuracy of the popular quadratic limb-darkening law \citep{2002ApJ...580L.171M} to within $\mathcal{O}(10^{-15})$ along with analytical derivatives. This study directly probes the speed and accuracy of various limb-darkening laws, and we seek to improve upon that work and study how different models affect the statistical inference of transit parameters.

The following packages which provided limb-darkened transit curves were tested: \texttt{exoplanet} \citep{2019ascl.soft10005F}, \texttt{BATMAN} \citep{2015PASP..127.1161K}, \texttt{PyTransit} \citep{Parviainen2015}, and \texttt{EXOFASTv2} \citep{2019arXiv190709480E}. These libraries have a variety of limb-darkening laws, but all have a quadratic limb-darkening law following the framework of \citet{2002ApJ...580L.171M}. In addition, \texttt{exoplanet} and \texttt{EXOFASTv2} have entire statistical modeling frameworks built into or on top of the transit curves. We also test \texttt{Juliet} \citep{2019MNRAS.490.2262E}, which adds a statistical modeling framework to the \texttt{BATMAN} transit models.

To begin, we used the highly accurate \citet{2020AJ....159..123A} transit models to simulate a light curve. We build a hierarchical model using the \texttt{exoplanet} framework with the following parameters: the semi-major axis, orbital period, time of inferior conjunction, ratio of planet to stellar radii, limb-darkening coefficients, and out-of-transit noise. The ground truth parameters were chosen to roughly mock the Kepler-101b transit \citep{2014A&A...572A...2B} and are shown in \autoref{table}.

\begin{deluxetable}{rCCCCCCC}
    \tablecaption{Posterior outputs from each statistical model experiment. The uncertainty bounds are given by the 68\% highest-posterior density interval. \label{table}}
    \tablehead{
        \colhead{} &
        \colhead{$P$ [d]} &
        \colhead{$t0$ [d]} &
        \colhead{$R_p/R_*$} &
        \colhead{$aR_*$} &
        \colhead{$u_1$} &
        \colhead{$u_2$} &
        \colhead{$\sigma$}
    }
    \startdata
    \textbf{ground truth} & 3.5 & 1.3 & 0.03 & 10 & 0.5 & 0.2 & 1e-4 \\
    exoplanet+NUTS & 3.5_{-5.81e-05}^{+5.66e-05} &
                     1.3_{-0.000181}^{+0.000186} &
                     0.0302_{-0.000104}^{+0.000119} &
                     10.0_{-0.0299}^{+0.0301} &
                     0.552_{-0.067}^{+0.0567} &
                     0.0479_{-0.101}^{+0.116} &
                     9.97e-05_{-5.52e-07}^{+6.15e-07} \\
    exoplanet+MH & 3.5_{-5.56e-05}^{+6.22e-05} &
                   1.3_{-0.000193}^{+0.000182} &
                   0.0302_{-0.00011}^{+0.000111} &
                   10.0_{-0.0296}^{+0.0306} &
                   0.552_{-0.0562}^{+0.0647} &
                   0.048_{-0.11}^{+0.103} &
                   9.97e-05_{-5.59e-07}^{+5.93e-07} \\
    BATMAN+MH & 3.5_{-5.28e-05}^{+6.32e-05} &
                1.3_{-0.000198}^{+0.000171} &
                0.0302_{-0.000118}^{+0.000106} &
                10.0_{-0.0275}^{+0.0337} &
                0.551_{-0.0694}^{+0.0568} &
                0.0523_{-0.106}^{+0.118} &
                9.97e-05_{-5.83e-07}^{+5.69e-07} \\
    PyTransit+MH & 3.5_{-5.45e-05}^{+5.95e-05} &
                   1.3_{-0.000168}^{+0.000194} &
                   0.0302_{-0.000109}^{+0.000115} &
                   10.0_{-0.027}^{+0.0336} &
                   0.555_{-0.0599}^{+0.0687} &
                   0.0422_{-0.116}^{+0.107} &
                   9.97e-05_{-5.64e-07}^{+5.95e-07} \\
    Juliet+NS & 3.5_{-5.67e-05}^{+5.83e-05} &
                1.3_{-0.000198}^{+0.000175} &
                0.0302_{-0.00011}^{+0.00011} &
                10.0_{-0.0325}^{+0.0273} &
                0.552_{-0.0572}^{+0.0632} &
                0.0476_{-0.108}^{+0.104} & \\
    EXOFASTv2+MH & 3.5_{-5.8e-05}^{+5.97e-05} &
                   1.3_{-0.000196}^{+0.000173} &
                   0.0302_{-9.79e-05}^{+0.000107} &
                   10.0_{-0.0249}^{+0.0227} &
                   0.405_{-0.0268}^{+0.0274} &
                   0.28_{-0.0403}^{+0.0411} &
                   9.98e-05_{-5.58e-07}^{+6.07e-07}
    \enddata
\end{deluxetable}
    

% \acknowledgments

% \software{
% astropy \citep{2013A&A...558A..33A,2018AJ....156..123A},
% numpy \citep{harris2020array},
% scikit-image \citep{2014arXiv1407.6245V},
% }

\bibliography{references}{}
\bibliographystyle{aasjournal}

\end{document}
